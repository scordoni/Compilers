%%%%%%%%%%%%%%%%%%%%%%%%%%%%%%%%%%%%%%%%%
%
% CMPT 432
% Spring 2022
% Lab Three
%
%%%%%%%%%%%%%%%%%%%%%%%%%%%%%%%%%%%%%%%%%

%%%%%%%%%%%%%%%%%%%%%%%%%%%%%%%%%%%%%%%%%
% Short Sectioned Assignment
% LaTeX Template
% Version 1.0 (5/5/12)
%
% This template has been downloaded from: http://www.LaTeXTemplates.com
% Original author: % Frits Wenneker (http://www.howtotex.com)
% License: CC BY-NC-SA 3.0 (http://creativecommons.org/licenses/by-nc-sa/3.0/)
% Modified by Alan G. Labouseur  - alan@labouseur.com
%
%%%%%%%%%%%%%%%%%%%%%%%%%%%%%%%%%%%%%%%%%

%----------------------------------------------------------------------------------------
%	PACKAGES AND OTHER DOCUMENT CONFIGURATIONS
%----------------------------------------------------------------------------------------

\documentclass[letterpaper, 10pt,DIV=13]{scrartcl} 

\usepackage{tikz}

\usepackage[T1]{fontenc} % Use 8-bit encoding that has 256 glyph
\usepackage[english]{babel} % English language/hyphenation
\usepackage{amsmath,amsfonts,amsthm,xfrac} % Math packages
\usepackage{sectsty} % Allows customizing section commands
\usepackage{graphicx}
\usepackage[lined,linesnumbered,commentsnumbered]{algorithm2e}
\usepackage{listings}
\usepackage{parskip}
\usepackage{lastpage}

\allsectionsfont{\normalfont\scshape} % Make all section titles in default font and small caps.

\usepackage{fancyhdr} % Custom headers and footers
\pagestyle{fancyplain} % Makes all pages in the document conform to the custom headers and footers

\fancyhead{} % No page header - if you want one, create it in the same way as the footers below
\fancyfoot[L]{} % Empty left footer
\fancyfoot[C]{} % Empty center footer
\fancyfoot[R]{page \thepage\ of \pageref{LastPage}} % Page numbering for right footer

\renewcommand{\headrulewidth}{0pt} % Remove header underlines
\renewcommand{\footrulewidth}{0pt} % Remove footer underlines
\setlength{\headheight}{13.6pt} % Customize the height of the header

\numberwithin{equation}{section} % Number equations within sections (i.e. 1.1, 1.2, 2.1, 2.2 instead of 1, 2, 3, 4)
\numberwithin{figure}{section} % Number figures within sections (i.e. 1.1, 1.2, 2.1, 2.2 instead of 1, 2, 3, 4)
\numberwithin{table}{section} % Number tables within sections (i.e. 1.1, 1.2, 2.1, 2.2 instead of 1, 2, 3, 4)

\setlength\parindent{0pt} % Removes all indentation from paragraphs.

\binoppenalty=3000
\relpenalty=3000

%----------------------------------------------------------------------------------------
%	TITLE SECTION
%----------------------------------------------------------------------------------------

\newcommand{\horrule}[1]{\rule{\linewidth}{#1}} % Create horizontal rule command with 1 argument of height

\title{	
   \normalfont \normalsize 
   \textsc{CMPT 432 - Spring 2022 - Dr. Labouseur} \\[10pt] % Header stuff.
   \horrule{0.5pt} \\[0.25cm] 	% Top horizontal rule
   \huge Lab Three  \\     	    % Assignment title
   \horrule{0.5pt} \\[0.25cm] 	% Bottom horizontal rule
}

\author{Shannon Cordoni \\ \normalsize Shannon.Cordoni@Marist.edu}

\date{\normalsize\today} 	% Today's date.

\begin{document}
\maketitle % Print the title

%----------------------------------------------------------------------------------------
%   start PROBLEM ONE
%----------------------------------------------------------------------------------------
\section{Problem One: Crafting a Compiler 4.7}

\paragraph{} Leftmost Derivation

    S. Start \\
    1. E\$ \\
    2. T plus E \$ \\
    2. T plus T plus E \$ \\
    4. T plus T times F plus E \$ \\
    3. F plus T times F plus E \$ \\
    3. F plus T times F plus T \$ \\
    5. F plus F times F plus T \$ \\
    5. F plus F times F plus F \$ \\
    7. num plus F times F plus F \$ \\
    7. num plus num times F plus F \$ \\
    7. num plus num times num plus F \$ \\
    7. num plus num times num plus num \$ 

\paragraph{} Rightmost Derivation

    S. Start \\
    1. E\$ \\
    2. T plus E \$ \\ 
    3. T plus T \$ \\ 
    4. T times F plus T times F \$ \\ 
    5. F times F plus T times F \$ \\ 
    5. F times F plus F times F \$ \\ 
    7. num times F plus F times F \$ \\ 
    7. num times num plus F times F \$ \\ 
    7. num times num plus num times F \$ \\ 
    7. num times num plus num times num \$ \\ 

\section{Problem Two: Crafting a Compiler 5.2c}

\paragraph{} The recursive descent parser would read:


\lstset{numbers=left, numberstyle=\tiny, stepnumber=1, numbersep=5pt, basicstyle=\footnotesize\ttfamily}
\begin{lstlisting}[frame=single, ] 

parseStart(){

    parseValue();
    match($);

}//parseStart

parseValue(){

    if(token.getName().compareToIgnoreCase("num"){
        match(num);
    }//if
    
    else if(token.getName().compareToIgnoreCase("Iparen"){
        match(Iparen);
        parseExpr();
        match(rparen);
    }//else if
    
}//parseValue

parseExpr(){

    if(token.getName().compareToIgnoreCase("plus"){
        match(plus);
        parseValue();
        parseValue();
    }//if
    
    else if(token.getName().compareToIgnoreCase("prod"){
        match(prod);
        parseValues();
    }//else if
    
}//parseExpr

parseValues(){

    if(token.getName().compareToIgnoreCase("Value"){
        parseValue();
        parseValues();
    }//if
    
    else{
        //for empty String
    }//else 
    
}//parseValues



\end{lstlisting}



\newpage

\section{Problem Three: Dragon 4.2.1}

\paragraph{} a) Leftmost Derivation\\

    S. S \\
    2. S S * \\
    1. S S + S *\\
    3. a S + S *\\
    3. a a + S *\\
    3. a a + a *\\


\paragraph{} b) Rightmost Derivation

    S. S \\
    2. S         S * \\
    1. S   S +   S *\\
    3. S   S +   a *\\
    3. S   a +   a *\\
    3. a   a +   a *\\

\paragraph{} c) Parse Tree

\begin{tikzpicture}
    
    \node {S} [sibling distance = 2.5cm]
    child {node {S} child {node {S} child {node {a}}} child {node {S} child {node {a}}} child {node {+}}}
    child {node {S} child {node [yshift = .5cm] {a}}}
    child {node {*}};

    
\end{tikzpicture}


%----------------------------------------------------------------------------------------
%   REFERENCES
%----------------------------------------------------------------------------------------
% The following two commands are all you need in the initial runs of your .Tex file to
% produce the bibliography for the citations in your paper.
\bibliographystyle{abbrv}
\bibliography{lab01} 
% You must have a proper ".bib" file and remember to run:
% latex bibtex latex latex
% to resolve all references.

\pagebreak
\end{document}