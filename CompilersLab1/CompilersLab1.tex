%%%%%%%%%%%%%%%%%%%%%%%%%%%%%%%%%%%%%%%%%
%
% CMPT 432
% Spring 2022
% Lab One
%
%%%%%%%%%%%%%%%%%%%%%%%%%%%%%%%%%%%%%%%%%

%%%%%%%%%%%%%%%%%%%%%%%%%%%%%%%%%%%%%%%%%
% Short Sectioned Assignment
% LaTeX Template
% Version 1.0 (5/5/12)
%
% This template has been downloaded from: http://www.LaTeXTemplates.com
% Original author: % Frits Wenneker (http://www.howtotex.com)
% License: CC BY-NC-SA 3.0 (http://creativecommons.org/licenses/by-nc-sa/3.0/)
% Modified by Alan G. Labouseur  - alan@labouseur.com
%
%%%%%%%%%%%%%%%%%%%%%%%%%%%%%%%%%%%%%%%%%

%----------------------------------------------------------------------------------------
%	PACKAGES AND OTHER DOCUMENT CONFIGURATIONS
%----------------------------------------------------------------------------------------

\documentclass[letterpaper, 10pt,DIV=13]{scrartcl} 

\usepackage[T1]{fontenc} % Use 8-bit encoding that has 256 glyph
\usepackage[english]{babel} % English language/hyphenation
\usepackage{amsmath,amsfonts,amsthm,xfrac} % Math packages
\usepackage{sectsty} % Allows customizing section commands
\usepackage{graphicx}
\usepackage[lined,linesnumbered,commentsnumbered]{algorithm2e}
\usepackage{listings}
\usepackage{parskip}
\usepackage{lastpage}

\allsectionsfont{\normalfont\scshape} % Make all section titles in default font and small caps.

\usepackage{fancyhdr} % Custom headers and footers
\pagestyle{fancyplain} % Makes all pages in the document conform to the custom headers and footers

\fancyhead{} % No page header - if you want one, create it in the same way as the footers below
\fancyfoot[L]{} % Empty left footer
\fancyfoot[C]{} % Empty center footer
\fancyfoot[R]{page \thepage\ of \pageref{LastPage}} % Page numbering for right footer

\renewcommand{\headrulewidth}{0pt} % Remove header underlines
\renewcommand{\footrulewidth}{0pt} % Remove footer underlines
\setlength{\headheight}{13.6pt} % Customize the height of the header

\numberwithin{equation}{section} % Number equations within sections (i.e. 1.1, 1.2, 2.1, 2.2 instead of 1, 2, 3, 4)
\numberwithin{figure}{section} % Number figures within sections (i.e. 1.1, 1.2, 2.1, 2.2 instead of 1, 2, 3, 4)
\numberwithin{table}{section} % Number tables within sections (i.e. 1.1, 1.2, 2.1, 2.2 instead of 1, 2, 3, 4)

\setlength\parindent{0pt} % Removes all indentation from paragraphs.

\binoppenalty=3000
\relpenalty=3000

%----------------------------------------------------------------------------------------
%	TITLE SECTION
%----------------------------------------------------------------------------------------

\newcommand{\horrule}[1]{\rule{\linewidth}{#1}} % Create horizontal rule command with 1 argument of height

\title{	
   \normalfont \normalsize 
   \textsc{CMPT 432 - Spring 2022 - Dr. Labouseur} \\[10pt] % Header stuff.
   \horrule{0.5pt} \\[0.25cm] 	% Top horizontal rule
   \huge Lab One  \\     	    % Assignment title
   \horrule{0.5pt} \\[0.25cm] 	% Bottom horizontal rule
}

\author{Shannon Cordoni \\ \normalsize Shannon.Cordoni@Marist.edu}

\date{\normalsize\today} 	% Today's date.

\begin{document}
\maketitle % Print the title

%----------------------------------------------------------------------------------------
%   start PROBLEM ONE
%----------------------------------------------------------------------------------------
\section{Problem One: Crafting a Compiler 1.11}

\paragraph{} MOSS is an interesting new technology used to find similarities in computer programs. It compares the similarities between to files and returns a percentage representing the amount of similarity between the two files.


\section{Problem Two: Crafting a Compiler 3.1}

\paragraph{} The token sequence that would be produced is:

\lstset{numbers=left, numberstyle=\tiny, stepnumber=1, numbersep=5pt, basicstyle=\footnotesize\ttfamily}
\begin{lstlisting}[frame=single, ] 

DEBUG Lexer - MAIN [ main ]
DEBUG Lexer - CONST [ const ]
DEBUG Lexer - FLOAT [ float ]
DEBUG Lexer - ID [ payment ]
DEBUG Lexer - ASSIGNMENT [ = ]
DEBUG Lexer - DIGIT [ 3 ]
DEBUG Lexer - DIGIT [ 8 ]
DEBUG Lexer - DIGIT [ 4 ]
DEBUG Lexer - DECIMAL [ . ]
DEBUG Lexer - DIGIT [ 0 ]
DEBUG Lexer - DIGIT [ 0 ]
DEBUG Lexer - SEMICOLON [ ; ]
DEBUG Lexer - FLOAT [ float ]
DEBUG Lexer - ID [ bal ]
DEBUG Lexer - SEMICOLON [ ; ]
DEBUG Lexer - INT [ int ]
DEBUG Lexer - ID [ month ]
DEBUG Lexer - ASSIGNMENT [ = ]
DEBUG Lexer - DIGIT [ 0 ]
DEBUG Lexer - SEMICOLON [ ; ]
DEBUG Lexer - ID [ bal ]
DEBUG Lexer - ASSIGNMENT [ = ]
DEBUG Lexer - DIGIT [ 1 ]
DEBUG Lexer - DIGIT [ 5 ]
DEBUG Lexer - DIGIT [ 0 ]
DEBUG Lexer - DIGIT [ 0 ]
DEBUG Lexer - DIGIT [ 0 ]
DEBUG Lexer - SEMICOLON [ ; ]
DEBUG Lexer - WHILE [ while ]
DEBUG Lexer - LPAREN [ ( ]
DEBUG Lexer - ID [ bal ]
DEBUG Lexer - GREATERTHAN [ > ]
DEBUG Lexer - DIGIT [ 0 ]
DEBUG Lexer - RPAREN [ ) ]
DEBUG Lexer - LBRACE [ { ]
DEBUG Lexer - PRINT [ print ]
DEBUG Lexer - LPAREN [ ( ]
DEBUG Lexer - OPENQUOTE [ " ]
DEBUG Lexer - CHAR [ M ]
DEBUG Lexer - CHAR [ o ]
DEBUG Lexer - CHAR [ n ]
DEBUG Lexer - CHAR [ t ]
DEBUG Lexer - CHAR [ h ]
DEBUG Lexer - SEMICOLON [ : ]
DEBUG Lexer - SPACE [  ]
DEBUG Lexer - PERCENT [ % ]
DEBUG Lexer - DIGIT [ 2 ]
DEBUG Lexer - CHAR [ d ]
DEBUG Lexer - SPACE [  ]
DEBUG Lexer - CHAR [ B ]
DEBUG Lexer - CHAR [ a ]
DEBUG Lexer - CHAR [ l ]
DEBUG Lexer - CHAR [ a ]
DEBUG Lexer - CHAR [ n ]
DEBUG Lexer - CHAR [ c ]
DEBUG Lexer - CHAR [ e ]
DEBUG Lexer - SEMICOLON [ : ]
DEBUG Lexer - SPACE [  ]
DEBUG Lexer - PERCENT [ % ]
DEBUG Lexer - DIGIT [ 1 ]
DEBUG Lexer - DIGIT [ 0 ]
DEBUG Lexer - DECIMAL [ . ]
DEBUG Lexer - DIGIT [ 2 ]
DEBUG Lexer - CHAR [ f ]
DEBUG Lexer - CHAR [ \ ]
DEBUG Lexer - CHAR [ n ]
DEBUG Lexer - CLOSEQUOTE [ " ]
DEBUG Lexer - COMMA [ , ]
DEBUG Lexer - ID [ month ]
DEBUG Lexer - COMMA [ , ]
DEBUG Lexer - ID [ bal ]
DEBUG Lexer - RPAREN [ ) ]
DEBUG Lexer - SEMICOLON [ ; ]
DEBUG Lexer - ID [ bal ]
DEBUG Lexer - ASSIGNMENT [ = ]
DEBUG Lexer - ID [ bal ]
DEBUG Lexer - MINUS [ - ]
DEBUG Lexer - ID [ payment ]
DEBUG Lexer - PLUS [ + ]
DEBUG Lexer - DIGIT [ 0 ]
DEBUG Lexer - DECIMAL [ . ]
DEBUG Lexer - DIGIT [ 0 ]
DEBUG Lexer - DIGIT [ 1 ]
DEBUG Lexer - DIGIT [ 5 ]
DEBUG Lexer - MULTIPLY [ * ]
DEBUG Lexer - ID [ bal ]
DEBUG Lexer - SEMICOLON [ ; ]
DEBUG Lexer - ID [ month ]
DEBUG Lexer - ASSIGNMENT [ = ]
DEBUG Lexer - ID [ month ]
DEBUG Lexer - PLUS [ + ]
DEBUG Lexer - DIGIT [ 1 ]
DEBUG Lexer - SEMICOLON [ ; ]
DEBUG Lexer - RBRACE [ } ]
DEBUG Lexer - RBRACE [ } ]
\end{lstlisting}

\section{Problem Three: Dragon 1.1.4}

The advantages to using C as a target language for a compiler is that is it available on many platforms allowing for wide spread use. Along with having a simple set of keywords it makes it easy for fast computation of machine code.

\section{Problem Three: Dragon 1.6.1}

The value assigned to $w$ is $i+j$, or $6+7$ which is 13. The value assigned to $x$ is also $i+j$, but this time it equivocates to $6+5$ which is 11. The value assigned to $y$ is also $i+j$, or $8+5$, which is 13. Lastly, the value assigned to $z$ which is also $i+j$, or $6+5$ which is 11.

%----------------------------------------------------------------------------------------
%   REFERENCES
%----------------------------------------------------------------------------------------
% The following two commands are all you need in the initial runs of your .Tex file to
% produce the bibliography for the citations in your paper.
\bibliographystyle{abbrv}
\bibliography{lab01} 
% You must have a proper ".bib" file and remember to run:
% latex bibtex latex latex
% to resolve all references.

\pagebreak
\end{document}